\documentclass[]{report}

\usepackage{amsmath,amsfonts}
\usepackage{graphicx}
\usepackage{url}
\usepackage[hidelinks]{hyperref}

% Title Page
\title{{\huge  TITLE} \\
	{\small Automatic Control\\
		Electronic Engineering for Intelligent Vehicles\\
		University of Bologna\\
		A.A. 202X-202X}}
\author{Student A and Student B and ...}


\begin{document}
	\maketitle
	
	\begin{abstract}
		Here briefly detail  the aims of the project.
	\end{abstract}
	
	\chapter{Introduction}
	
\subsection{System Linearization}

To facilitate the control design of the longitudinal half-car model equipped with active front and rear suspension systems, the initial step involves the linearization of the nonlinear system dynamics. This process is performed by identifying appropriate steady-state operating points \((x^*, y^*, w^*)\), which characterize representative conditions under which the vehicle is expected to operate. Linearizing the system around these equilibrium points enables the derivation of a time-invariant linear approximation of the vehicle dynamics, thereby simplifying the synthesis and analysis of control strategies.

\subsubsection{Linearization Around the Operating Point}

Consider the nonlinear system model:
\begin{equation}
	\label{eq:nonlinear_model}
	\begin{aligned}
		\dot{x} &= f(x, u, w), \quad x(t_0) = x_0 \\
		y &= h(x, u, w) \\
		e &= h_e(x, u, w)
	\end{aligned}
\end{equation}

The steady-state operating points \((x^\star, u^\star, w^\star)\) is called \textit{equilibrium triplet} if satisfies the condition:
\begin{equation}
	\label{eq:equilibrium_triplet}
	f(x^\star, u^\star, w^\star) = 0
\end{equation}
and defines the equilibrium output and error as:
\begin{equation}
	\label{eq:equilibrium_output_error}
	y^\star := h(x^\star, u^\star, w^\star), \quad e^\star := h_e(x^\star, u^\star, w^\star)
\end{equation}

The variations around the equilibrium point are defined as:
\begin{equation}
	\label{eq:variations}
	\begin{aligned}
		\tilde{x} &:= x - x^\star \\
		\tilde{y} &:= y - y^\star \\
		\tilde{e} &:= e - e^\star \\
		\tilde{u} &:= u - u^\star \\
		\tilde{w} &:= w - w^\star \\
	\end{aligned}
\end{equation}

Using the fact that \(\dot{x}^\star = 0\), the dynamics of the variations are:
\begin{equation}
	\label{eq:variation_dynamics}
	\begin{aligned}
		\dot{\tilde{x}} &= f(x^\star + \tilde{x}, u^\star + \tilde{u}, w^\star + \tilde{w}), \quad \tilde{x}(t_0) = x_0 - x^\star \\
		\tilde{y} &= h(x^\star + \tilde{x}, u^\star + \tilde{u}, w^\star + \tilde{w}) \\
		\tilde{e} &= h_e(x^\star + \tilde{x}, u^\star + \tilde{u}, w^\star + \tilde{w})
	\end{aligned}
\end{equation}


To obtain a tractable model for controller synthesis, we apply a first-order Taylor expansion around the equilibrium point. The resulting Jacobian matrices are defined as:
\begin{equation}
	\label{eq:jacobians_compact_matrix}
	\begin{aligned}
		A &:= \left. \frac{\partial f(x, u, w)}{\partial x} \right|_{\substack{x = x^\star \\ u = u^\star \\ w = w^\star}} & B_1 &:= \left. \frac{\partial f(x, u, w)}{\partial u} \right|_{\substack{x = x^\star \\ u = u^\star \\ w = w^\star}} & B_2 &:= \left. \frac{\partial f(x, u, w)}{\partial w} \right|_{\substack{x = x^\star \\ u = u^\star \\ w = w^\star}} \\
		C &:= \left. \frac{\partial h(x, u, w)}{\partial x} \right|_{\substack{x = x^\star \\ u = u^\star \\ w = w^\star}} & D_1 &:= \left. \frac{\partial h(x, u, w)}{\partial u} \right|_{\substack{x = x^\star \\ u = u^\star \\ w = w^\star}} & D_2 &:= \left. \frac{\partial h(x, u, w)}{\partial w} \right|_{\substack{x = x^\star \\ u = u^\star \\ w = w^\star}} \\
		C_e &:= \left. \frac{\partial h_e(x, u, w)}{\partial x} \right|_{\substack{x = x^\star \\ u = u^\star \\ w = w^\star}} & D_{e1} &:= \left. \frac{\partial h_e(x, u, w)}{\partial u} \right|_{\substack{x = x^\star \\ u = u^\star \\ w = w^\star}} & D_{e2} &:= \left. \frac{\partial h_e(x, u, w)}{\partial w} \right|_{\substack{x = x^\star \\ u = u^\star \\ w = w^\star}}
	\end{aligned}
\end{equation}





Neglecting second-order terms, the linearized system becomes the so-called \textit{design model}:
\begin{equation}
\label{eq:linearized_system}
\left\{
\begin{aligned}
	\dot{\tilde{x}} &= A \tilde{x} + B_1 \tilde{u} + B_2 \tilde{w}, \quad \tilde{x}(t_0) = x_0 - x^\star \\
	\tilde{y} &= C \tilde{x} + D_1 \tilde{u} + D_2 \tilde{w} \\
	\tilde{e} &= C_e \tilde{x} + D_{e1} \tilde{u} + D_{e2} \tilde{w}
\end{aligned}
\right.
\end{equation}


This Linear Time-Invariant (LTI) approximation of the nonlinear model is valid in a neighborhood of the equilibrium point, enabling efficient analysis and controller design under small perturbations.\\

 
	
		
\textbf{TODO}: Inserire le matrici calcolate con matlab e verificare che rispettino quello che abbiamo scritto (verifica che il codice del .m e quello che fa rispetta queste cose dette nella aprte di teoria)		
		
		
		
	
	
	
	
\end{document}          
